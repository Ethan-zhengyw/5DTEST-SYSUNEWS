% Generated by Sphinx.
\def\sphinxdocclass{report}
\documentclass[letterpaper,10pt,english]{sphinxmanual}
\usepackage[utf8]{inputenc}
\DeclareUnicodeCharacter{00A0}{\nobreakspace}
\usepackage{cmap}
\usepackage[T1]{fontenc}
\usepackage{babel}
\usepackage{times}
\usepackage[Bjarne]{fncychap}
\usepackage{longtable}
\usepackage{sphinx}
\usepackage{multirow}


\title{src Documentation}
\date{November 29, 2014}
\release{}
\author{Yiwang ZHENG}
\newcommand{\sphinxlogo}{}
\renewcommand{\releasename}{Release}
\makeindex

\makeatletter
\def\PYG@reset{\let\PYG@it=\relax \let\PYG@bf=\relax%
    \let\PYG@ul=\relax \let\PYG@tc=\relax%
    \let\PYG@bc=\relax \let\PYG@ff=\relax}
\def\PYG@tok#1{\csname PYG@tok@#1\endcsname}
\def\PYG@toks#1+{\ifx\relax#1\empty\else%
    \PYG@tok{#1}\expandafter\PYG@toks\fi}
\def\PYG@do#1{\PYG@bc{\PYG@tc{\PYG@ul{%
    \PYG@it{\PYG@bf{\PYG@ff{#1}}}}}}}
\def\PYG#1#2{\PYG@reset\PYG@toks#1+\relax+\PYG@do{#2}}

\expandafter\def\csname PYG@tok@gd\endcsname{\def\PYG@tc##1{\textcolor[rgb]{0.63,0.00,0.00}{##1}}}
\expandafter\def\csname PYG@tok@gu\endcsname{\let\PYG@bf=\textbf\def\PYG@tc##1{\textcolor[rgb]{0.50,0.00,0.50}{##1}}}
\expandafter\def\csname PYG@tok@gt\endcsname{\def\PYG@tc##1{\textcolor[rgb]{0.00,0.27,0.87}{##1}}}
\expandafter\def\csname PYG@tok@gs\endcsname{\let\PYG@bf=\textbf}
\expandafter\def\csname PYG@tok@gr\endcsname{\def\PYG@tc##1{\textcolor[rgb]{1.00,0.00,0.00}{##1}}}
\expandafter\def\csname PYG@tok@cm\endcsname{\let\PYG@it=\textit\def\PYG@tc##1{\textcolor[rgb]{0.25,0.50,0.56}{##1}}}
\expandafter\def\csname PYG@tok@vg\endcsname{\def\PYG@tc##1{\textcolor[rgb]{0.73,0.38,0.84}{##1}}}
\expandafter\def\csname PYG@tok@m\endcsname{\def\PYG@tc##1{\textcolor[rgb]{0.13,0.50,0.31}{##1}}}
\expandafter\def\csname PYG@tok@mh\endcsname{\def\PYG@tc##1{\textcolor[rgb]{0.13,0.50,0.31}{##1}}}
\expandafter\def\csname PYG@tok@cs\endcsname{\def\PYG@tc##1{\textcolor[rgb]{0.25,0.50,0.56}{##1}}\def\PYG@bc##1{\setlength{\fboxsep}{0pt}\colorbox[rgb]{1.00,0.94,0.94}{\strut ##1}}}
\expandafter\def\csname PYG@tok@ge\endcsname{\let\PYG@it=\textit}
\expandafter\def\csname PYG@tok@vc\endcsname{\def\PYG@tc##1{\textcolor[rgb]{0.73,0.38,0.84}{##1}}}
\expandafter\def\csname PYG@tok@il\endcsname{\def\PYG@tc##1{\textcolor[rgb]{0.13,0.50,0.31}{##1}}}
\expandafter\def\csname PYG@tok@go\endcsname{\def\PYG@tc##1{\textcolor[rgb]{0.20,0.20,0.20}{##1}}}
\expandafter\def\csname PYG@tok@cp\endcsname{\def\PYG@tc##1{\textcolor[rgb]{0.00,0.44,0.13}{##1}}}
\expandafter\def\csname PYG@tok@gi\endcsname{\def\PYG@tc##1{\textcolor[rgb]{0.00,0.63,0.00}{##1}}}
\expandafter\def\csname PYG@tok@gh\endcsname{\let\PYG@bf=\textbf\def\PYG@tc##1{\textcolor[rgb]{0.00,0.00,0.50}{##1}}}
\expandafter\def\csname PYG@tok@ni\endcsname{\let\PYG@bf=\textbf\def\PYG@tc##1{\textcolor[rgb]{0.84,0.33,0.22}{##1}}}
\expandafter\def\csname PYG@tok@nl\endcsname{\let\PYG@bf=\textbf\def\PYG@tc##1{\textcolor[rgb]{0.00,0.13,0.44}{##1}}}
\expandafter\def\csname PYG@tok@nn\endcsname{\let\PYG@bf=\textbf\def\PYG@tc##1{\textcolor[rgb]{0.05,0.52,0.71}{##1}}}
\expandafter\def\csname PYG@tok@no\endcsname{\def\PYG@tc##1{\textcolor[rgb]{0.38,0.68,0.84}{##1}}}
\expandafter\def\csname PYG@tok@na\endcsname{\def\PYG@tc##1{\textcolor[rgb]{0.25,0.44,0.63}{##1}}}
\expandafter\def\csname PYG@tok@nb\endcsname{\def\PYG@tc##1{\textcolor[rgb]{0.00,0.44,0.13}{##1}}}
\expandafter\def\csname PYG@tok@nc\endcsname{\let\PYG@bf=\textbf\def\PYG@tc##1{\textcolor[rgb]{0.05,0.52,0.71}{##1}}}
\expandafter\def\csname PYG@tok@nd\endcsname{\let\PYG@bf=\textbf\def\PYG@tc##1{\textcolor[rgb]{0.33,0.33,0.33}{##1}}}
\expandafter\def\csname PYG@tok@ne\endcsname{\def\PYG@tc##1{\textcolor[rgb]{0.00,0.44,0.13}{##1}}}
\expandafter\def\csname PYG@tok@nf\endcsname{\def\PYG@tc##1{\textcolor[rgb]{0.02,0.16,0.49}{##1}}}
\expandafter\def\csname PYG@tok@si\endcsname{\let\PYG@it=\textit\def\PYG@tc##1{\textcolor[rgb]{0.44,0.63,0.82}{##1}}}
\expandafter\def\csname PYG@tok@s2\endcsname{\def\PYG@tc##1{\textcolor[rgb]{0.25,0.44,0.63}{##1}}}
\expandafter\def\csname PYG@tok@vi\endcsname{\def\PYG@tc##1{\textcolor[rgb]{0.73,0.38,0.84}{##1}}}
\expandafter\def\csname PYG@tok@nt\endcsname{\let\PYG@bf=\textbf\def\PYG@tc##1{\textcolor[rgb]{0.02,0.16,0.45}{##1}}}
\expandafter\def\csname PYG@tok@nv\endcsname{\def\PYG@tc##1{\textcolor[rgb]{0.73,0.38,0.84}{##1}}}
\expandafter\def\csname PYG@tok@s1\endcsname{\def\PYG@tc##1{\textcolor[rgb]{0.25,0.44,0.63}{##1}}}
\expandafter\def\csname PYG@tok@gp\endcsname{\let\PYG@bf=\textbf\def\PYG@tc##1{\textcolor[rgb]{0.78,0.36,0.04}{##1}}}
\expandafter\def\csname PYG@tok@sh\endcsname{\def\PYG@tc##1{\textcolor[rgb]{0.25,0.44,0.63}{##1}}}
\expandafter\def\csname PYG@tok@ow\endcsname{\let\PYG@bf=\textbf\def\PYG@tc##1{\textcolor[rgb]{0.00,0.44,0.13}{##1}}}
\expandafter\def\csname PYG@tok@sx\endcsname{\def\PYG@tc##1{\textcolor[rgb]{0.78,0.36,0.04}{##1}}}
\expandafter\def\csname PYG@tok@bp\endcsname{\def\PYG@tc##1{\textcolor[rgb]{0.00,0.44,0.13}{##1}}}
\expandafter\def\csname PYG@tok@c1\endcsname{\let\PYG@it=\textit\def\PYG@tc##1{\textcolor[rgb]{0.25,0.50,0.56}{##1}}}
\expandafter\def\csname PYG@tok@kc\endcsname{\let\PYG@bf=\textbf\def\PYG@tc##1{\textcolor[rgb]{0.00,0.44,0.13}{##1}}}
\expandafter\def\csname PYG@tok@c\endcsname{\let\PYG@it=\textit\def\PYG@tc##1{\textcolor[rgb]{0.25,0.50,0.56}{##1}}}
\expandafter\def\csname PYG@tok@mf\endcsname{\def\PYG@tc##1{\textcolor[rgb]{0.13,0.50,0.31}{##1}}}
\expandafter\def\csname PYG@tok@err\endcsname{\def\PYG@bc##1{\setlength{\fboxsep}{0pt}\fcolorbox[rgb]{1.00,0.00,0.00}{1,1,1}{\strut ##1}}}
\expandafter\def\csname PYG@tok@kd\endcsname{\let\PYG@bf=\textbf\def\PYG@tc##1{\textcolor[rgb]{0.00,0.44,0.13}{##1}}}
\expandafter\def\csname PYG@tok@ss\endcsname{\def\PYG@tc##1{\textcolor[rgb]{0.32,0.47,0.09}{##1}}}
\expandafter\def\csname PYG@tok@sr\endcsname{\def\PYG@tc##1{\textcolor[rgb]{0.14,0.33,0.53}{##1}}}
\expandafter\def\csname PYG@tok@mo\endcsname{\def\PYG@tc##1{\textcolor[rgb]{0.13,0.50,0.31}{##1}}}
\expandafter\def\csname PYG@tok@mi\endcsname{\def\PYG@tc##1{\textcolor[rgb]{0.13,0.50,0.31}{##1}}}
\expandafter\def\csname PYG@tok@kn\endcsname{\let\PYG@bf=\textbf\def\PYG@tc##1{\textcolor[rgb]{0.00,0.44,0.13}{##1}}}
\expandafter\def\csname PYG@tok@o\endcsname{\def\PYG@tc##1{\textcolor[rgb]{0.40,0.40,0.40}{##1}}}
\expandafter\def\csname PYG@tok@kr\endcsname{\let\PYG@bf=\textbf\def\PYG@tc##1{\textcolor[rgb]{0.00,0.44,0.13}{##1}}}
\expandafter\def\csname PYG@tok@s\endcsname{\def\PYG@tc##1{\textcolor[rgb]{0.25,0.44,0.63}{##1}}}
\expandafter\def\csname PYG@tok@kp\endcsname{\def\PYG@tc##1{\textcolor[rgb]{0.00,0.44,0.13}{##1}}}
\expandafter\def\csname PYG@tok@w\endcsname{\def\PYG@tc##1{\textcolor[rgb]{0.73,0.73,0.73}{##1}}}
\expandafter\def\csname PYG@tok@kt\endcsname{\def\PYG@tc##1{\textcolor[rgb]{0.56,0.13,0.00}{##1}}}
\expandafter\def\csname PYG@tok@sc\endcsname{\def\PYG@tc##1{\textcolor[rgb]{0.25,0.44,0.63}{##1}}}
\expandafter\def\csname PYG@tok@sb\endcsname{\def\PYG@tc##1{\textcolor[rgb]{0.25,0.44,0.63}{##1}}}
\expandafter\def\csname PYG@tok@k\endcsname{\let\PYG@bf=\textbf\def\PYG@tc##1{\textcolor[rgb]{0.00,0.44,0.13}{##1}}}
\expandafter\def\csname PYG@tok@se\endcsname{\let\PYG@bf=\textbf\def\PYG@tc##1{\textcolor[rgb]{0.25,0.44,0.63}{##1}}}
\expandafter\def\csname PYG@tok@sd\endcsname{\let\PYG@it=\textit\def\PYG@tc##1{\textcolor[rgb]{0.25,0.44,0.63}{##1}}}

\def\PYGZbs{\char`\\}
\def\PYGZus{\char`\_}
\def\PYGZob{\char`\{}
\def\PYGZcb{\char`\}}
\def\PYGZca{\char`\^}
\def\PYGZam{\char`\&}
\def\PYGZlt{\char`\<}
\def\PYGZgt{\char`\>}
\def\PYGZsh{\char`\#}
\def\PYGZpc{\char`\%}
\def\PYGZdl{\char`\$}
\def\PYGZhy{\char`\-}
\def\PYGZsq{\char`\'}
\def\PYGZdq{\char`\"}
\def\PYGZti{\char`\~}
% for compatibility with earlier versions
\def\PYGZat{@}
\def\PYGZlb{[}
\def\PYGZrb{]}
\makeatother

\begin{document}

\maketitle
\tableofcontents
\phantomsection\label{index::doc}


Contents:


\chapter{sysunews package}
\label{sysunews::doc}\label{sysunews:welcome-to-sysunews-s-documentation}\label{sysunews:sysunews-package}

\section{Submodules}
\label{sysunews:submodules}

\section{sysunews.api module}
\label{sysunews:sysunews-api-module}\label{sysunews:module-sysunews.api}\index{sysunews.api (module)}
api module, call by db\_update module mainly

Function in this module mainly call function in html\_extracting module to finish their job.
\index{get\_index\_range() (in module sysunews.api)}

\begin{fulllineitems}
\phantomsection\label{sysunews:sysunews.api.get_index_range}\pysiglinewithargsret{\code{sysunews.api.}\bfcode{get\_index\_range}}{\emph{data}}{}
find the index range
\begin{quote}\begin{description}
\item[{Parameters}] \leavevmode
\textbf{data} -- content of {[}\href{http://news2.sysu.edu.cn/news0*/index.htm}{http://news2.sysu.edu.cn/news0*/index.htm}{]}

\item[{Returns}] \leavevmode
\begin{itemize}
\item {} 
\emph{start}: an integer, corresponding to the start of index

\item {} 
\emph{end}: an integer, corresponding to the end of index

\end{itemize}


\end{description}\end{quote}

\end{fulllineitems}

\index{get\_module() (in module sysunews.api)}

\begin{fulllineitems}
\phantomsection\label{sysunews:sysunews.api.get_module}\pysiglinewithargsret{\code{sysunews.api.}\bfcode{get\_module}}{\emph{url}}{}
Get the module id of a news from its url
\begin{quote}\begin{description}
\item[{Parameters}] \leavevmode
\textbf{url} -- news' url

\item[{Returns}] \leavevmode
module - the module id

\end{description}\end{quote}

\end{fulllineitems}

\index{get\_news() (in module sysunews.api)}

\begin{fulllineitems}
\phantomsection\label{sysunews:sysunews.api.get_news}\pysiglinewithargsret{\code{sysunews.api.}\bfcode{get\_news}}{\emph{news\_url}}{}
Get a news from its url
\begin{quote}\begin{description}
\item[{Parameters}] \leavevmode
\textbf{url} -- news' url

\item[{Returns}] \leavevmode
news - a dictionary store many attrubites of the news in key-value

\end{description}\end{quote}

\end{fulllineitems}

\index{get\_news\_fromDB() (in module sysunews.api)}

\begin{fulllineitems}
\phantomsection\label{sysunews:sysunews.api.get_news_fromDB}\pysiglinewithargsret{\code{sysunews.api.}\bfcode{get\_news\_fromDB}}{\emph{module}, \emph{start=1}, \emph{num=-1}}{}
Query data from database

When querying news from DB, the news will be return according to the order of update time, which means the start 1 is the newest news
\begin{quote}\begin{description}
\item[{Parameters}] \leavevmode\begin{itemize}
\item {} 
\textbf{module} -- an integer to point out from which module to find news

\item {} 
\textbf{start} -- the start index in DB, 1 means the newest one

\item {} 
\textbf{num} -- how many news want to get

\end{itemize}

\item[{Returns}] \leavevmode
list - a list of news, every news is a dictionary

\end{description}\end{quote}

\end{fulllineitems}

\index{get\_news\_urls() (in module sysunews.api)}

\begin{fulllineitems}
\phantomsection\label{sysunews:sysunews.api.get_news_urls}\pysiglinewithargsret{\code{sysunews.api.}\bfcode{get\_news\_urls}}{\emph{module}}{}
Get all urls of certain module
\begin{quote}\begin{description}
\item[{Parameters}] \leavevmode
\textbf{module} -- urls of which module to get

\item[{Returns}] \leavevmode
urls - a list of urls

\end{description}\end{quote}

\end{fulllineitems}

\index{get\_newsid() (in module sysunews.api)}

\begin{fulllineitems}
\phantomsection\label{sysunews:sysunews.api.get_newsid}\pysiglinewithargsret{\code{sysunews.api.}\bfcode{get\_newsid}}{\emph{url}}{}
Get the newsid of a news from its url
\begin{quote}\begin{description}
\item[{Parameters}] \leavevmode
\textbf{url} -- news' url

\item[{Returns}] \leavevmode
newsid - str of the newsid

\end{description}\end{quote}

\end{fulllineitems}

\index{get\_newsid\_firstpage() (in module sysunews.api)}

\begin{fulllineitems}
\phantomsection\label{sysunews:sysunews.api.get_newsid_firstpage}\pysiglinewithargsret{\code{sysunews.api.}\bfcode{get\_newsid\_firstpage}}{\emph{module}}{}
find the newsid

get the newsid in the first page of three module 
in order to check whether some new news has been post
\begin{quote}\begin{description}
\item[{Parameters}] \leavevmode
\textbf{module} -- module to get, range form 1 to 3

\item[{Returns}] \leavevmode
\emph{newsids} - a list of newsid will be returned

\end{description}\end{quote}

\end{fulllineitems}



\section{sysunews.db\_module module}
\label{sysunews:sysunews-db-module-module}\label{sysunews:module-sysunews.db_module}\index{sysunews.db\_module (module)}
Module to process database
\index{check\_news() (in module sysunews.db\_module)}

\begin{fulllineitems}
\phantomsection\label{sysunews:sysunews.db_module.check_news}\pysiglinewithargsret{\code{sysunews.db\_module.}\bfcode{check\_news}}{\emph{newsid}}{}
Check whether the news already exist in table urls
\begin{quote}\begin{description}
\item[{Parameters}] \leavevmode
\textbf{tablename} -- check the news in which table

\item[{Returns}] \leavevmode
return \emph{True} if find new with the same newsid in table urls else return \emph{False}

\end{description}\end{quote}

\end{fulllineitems}

\index{cleandb() (in module sysunews.db\_module)}

\begin{fulllineitems}
\phantomsection\label{sysunews:sysunews.db_module.cleandb}\pysiglinewithargsret{\code{sysunews.db\_module.}\bfcode{cleandb}}{}{}
Clean the database

This fuction will delete two table: urls and news in sysunewsDB

\end{fulllineitems}

\index{get\_module\_newsNum() (in module sysunews.db\_module)}

\begin{fulllineitems}
\phantomsection\label{sysunews:sysunews.db_module.get_module_newsNum}\pysiglinewithargsret{\code{sysunews.db\_module.}\bfcode{get\_module\_newsNum}}{\emph{module}}{}
get the num of news is certain module
\begin{quote}\begin{description}
\item[{Parameters}] \leavevmode
\textbf{module} -- find news number of which module

\item[{Returns}] \leavevmode
\emph{int} - the number of news of certain module in database

\end{description}\end{quote}

\end{fulllineitems}

\index{get\_news() (in module sysunews.db\_module)}

\begin{fulllineitems}
\phantomsection\label{sysunews:sysunews.db_module.get_news}\pysiglinewithargsret{\code{sysunews.db\_module.}\bfcode{get\_news}}{\emph{module}, \emph{start}, \emph{num}}{}
Query data from database

When querying news from DB, the news will be return according to the order of update time, which means the start 1 is the newest news
\begin{quote}\begin{description}
\item[{Parameters}] \leavevmode\begin{itemize}
\item {} 
\textbf{module} -- find news from which module

\item {} 
\textbf{start} -- the start index in DB

\item {} 
\textbf{num} -- how many news want to get

\end{itemize}

\item[{Returns}] \leavevmode
\emph{result} - a list of news, every news is a dictionary

\end{description}\end{quote}

\end{fulllineitems}

\index{resave\_img() (in module sysunews.db\_module)}

\begin{fulllineitems}
\phantomsection\label{sysunews:sysunews.db_module.resave_img}\pysiglinewithargsret{\code{sysunews.db\_module.}\bfcode{resave\_img}}{\emph{url}}{}
overwrite the image

When images was broken in local host, the images will be resave in directory /home/images/, and it will not check whether it's already exists
\begin{quote}\begin{description}
\item[{Parameters}] \leavevmode
\textbf{url} -- the url of the image, like {[}images/content/2014-11/20141124172306071544.jpg{]}

\end{description}\end{quote}

\end{fulllineitems}

\index{save\_img() (in module sysunews.db\_module)}

\begin{fulllineitems}
\phantomsection\label{sysunews:sysunews.db_module.save_img}\pysiglinewithargsret{\code{sysunews.db\_module.}\bfcode{save\_img}}{\emph{url}}{}
Save the image

The images will be save in directory /home/images/, and it will check whether it's already exists
\begin{quote}\begin{description}
\item[{Parameters}] \leavevmode
\textbf{url} -- the url of the image, like {[}images/content/2014-11/20141124172306071544.jpg{]}

\end{description}\end{quote}

\end{fulllineitems}

\index{save\_news() (in module sysunews.db\_module)}

\begin{fulllineitems}
\phantomsection\label{sysunews:sysunews.db_module.save_news}\pysiglinewithargsret{\code{sysunews.db\_module.}\bfcode{save\_news}}{\emph{news}}{}
Save news into database

Will check whether the news is already in database
\begin{quote}\begin{description}
\item[{Parameters}] \leavevmode
\textbf{news} -- a news dictionary

\end{description}\end{quote}

\end{fulllineitems}

\index{save\_urls() (in module sysunews.db\_module)}

\begin{fulllineitems}
\phantomsection\label{sysunews:sysunews.db_module.save_urls}\pysiglinewithargsret{\code{sysunews.db\_module.}\bfcode{save\_urls}}{\emph{urls}}{}
Save the urls into database

When url is already in database, it will just ignore it quietly
\begin{quote}\begin{description}
\item[{Parameters}] \leavevmode
\textbf{urls} -- url list

\end{description}\end{quote}

\end{fulllineitems}



\section{sysunews.db\_update module}
\label{sysunews:sysunews-db-update-module}\label{sysunews:module-sysunews.db_update}\index{sysunews.db\_update (module)}\index{initial() (in module sysunews.db\_update)}

\begin{fulllineitems}
\phantomsection\label{sysunews:sysunews.db_update.initial}\pysiglinewithargsret{\code{sysunews.db\_update.}\bfcode{initial}}{}{}
Initialize the database

::
This function will get all possible news url from the website and save them into database when it's not exist
Should be call by user when it's first time to setup:

\begin{Verbatim}[commandchars=\\\{\}]
sudo python db\PYGZus{}initial.py
\end{Verbatim}

\end{fulllineitems}

\index{update() (in module sysunews.db\_update)}

\begin{fulllineitems}
\phantomsection\label{sysunews:sysunews.db_update.update}\pysiglinewithargsret{\code{sysunews.db\_update.}\bfcode{update}}{}{}
update the database

This function will be use in server.py to check whether there is new news post into the website every 10 minutes and save new news into database,

\end{fulllineitems}



\section{sysunews.html\_extracting module}
\label{sysunews:module-sysunews.html_extracting}\label{sysunews:sysunews-html-extracting-module}\index{sysunews.html\_extracting (module)}
HTML content processing

This file is the module to processing html content, using regular expression module to match the target content and return them to function in api mainly.
\index{find\_index\_range() (in module sysunews.html\_extracting)}

\begin{fulllineitems}
\phantomsection\label{sysunews:sysunews.html_extracting.find_index_range}\pysiglinewithargsret{\code{sysunews.html\_extracting.}\bfcode{find\_index\_range}}{\emph{data}}{}
find the index range
\begin{quote}\begin{description}
\item[{Parameters}] \leavevmode
\textbf{data} -- content of {[}\href{http://news2.sysu.edu.cn/news0*/index.htm}{http://news2.sysu.edu.cn/news0*/index.htm}{]}

\item[{Returns}] \leavevmode
\begin{itemize}
\item {} 
\emph{start}: an integer, corresponding to the start of index

\item {} 
\emph{end}: an integer, corresponding to the end of index

\end{itemize}


\end{description}\end{quote}

\end{fulllineitems}

\index{find\_module() (in module sysunews.html\_extracting)}

\begin{fulllineitems}
\phantomsection\label{sysunews:sysunews.html_extracting.find_module}\pysiglinewithargsret{\code{sysunews.html\_extracting.}\bfcode{find\_module}}{\emph{url}}{}
Get the module id of a news from its url
\begin{quote}\begin{description}
\item[{Parameters}] \leavevmode
\textbf{url} -- news' url

\item[{Returns}] \leavevmode
\emph{module} - the module id of the news in url

\end{description}\end{quote}

\end{fulllineitems}

\index{find\_news() (in module sysunews.html\_extracting)}

\begin{fulllineitems}
\phantomsection\label{sysunews:sysunews.html_extracting.find_news}\pysiglinewithargsret{\code{sysunews.html\_extracting.}\bfcode{find\_news}}{\emph{data}}{}
Extracting the news attributes from data
\begin{quote}\begin{description}
\item[{Parameters}] \leavevmode
\textbf{data} -- the conten of a news html page

\item[{Returns}] \leavevmode
\emph{news} - return a structed news

\end{description}\end{quote}

\end{fulllineitems}

\index{find\_news\_urls() (in module sysunews.html\_extracting)}

\begin{fulllineitems}
\phantomsection\label{sysunews:sysunews.html_extracting.find_news_urls}\pysiglinewithargsret{\code{sysunews.html\_extracting.}\bfcode{find\_news\_urls}}{\emph{data}, \emph{module}}{}
Find news url in data
\begin{quote}\begin{description}
\item[{Parameters}] \leavevmode
\textbf{data} -- content of a page ex.{[}\href{http://news2.sysu.edu.cn/news01/index1.htm}{http://news2.sysu.edu.cn/news01/index1.htm}{]}

\item[{Returns}] \leavevmode
\emph{list} - a url list

\end{description}\end{quote}

\end{fulllineitems}

\index{find\_newsid() (in module sysunews.html\_extracting)}

\begin{fulllineitems}
\phantomsection\label{sysunews:sysunews.html_extracting.find_newsid}\pysiglinewithargsret{\code{sysunews.html\_extracting.}\bfcode{find\_newsid}}{\emph{url}}{}
Get the newsid of a news from its url
\begin{quote}\begin{description}
\item[{Parameters}] \leavevmode
\textbf{url} -- news' url

\item[{Returns}] \leavevmode
\emph{newsid} - the newsid of the news in url

\end{description}\end{quote}

\end{fulllineitems}



\section{sysunews.initial module}
\label{sysunews:sysunews-initial-module}\label{sysunews:module-sysunews.initial}\index{sysunews.initial (module)}
initialize the database

::
When it's the first time to deploy this server program, please enter directory sysunews/,
and run this follow two commands:

\begin{Verbatim}[commandchars=\\\{\}]
mysql \PYGZhy{}u root \PYGZhy{}p
source db\PYGZus{}initial.sql
\end{Verbatim}

::
after this two commands, a database sysunewsDB with two table (urls \& news) will have been created in your mysql, and then execute:

\begin{Verbatim}[commandchars=\\\{\}]
python initial.py
\end{Verbatim}

This will crawl news from news.sysu.edu.cn and save them into the database created before


\section{sysunews.post\_module module}
\label{sysunews:module-sysunews.post_module}\label{sysunews:sysunews-post-module-module}\index{sysunews.post\_module (module)}
Main Request object

This file contains some definition of Request object to be used in api to get the content of certain website page
\index{req\_easy\_req() (in module sysunews.post\_module)}

\begin{fulllineitems}
\phantomsection\label{sysunews:sysunews.post_module.req_easy_req}\pysiglinewithargsret{\code{sysunews.post\_module.}\bfcode{req\_easy\_req}}{\emph{address}}{}
get a simple request object
\begin{quote}\begin{description}
\item[{Parameters}] \leavevmode
\textbf{address} -- the url to open

\item[{Returns}] \leavevmode
\emph{Request} - a Request object

\end{description}\end{quote}

\end{fulllineitems}

\index{req\_get\_news() (in module sysunews.post\_module)}

\begin{fulllineitems}
\phantomsection\label{sysunews:sysunews.post_module.req_get_news}\pysiglinewithargsret{\code{sysunews.post\_module.}\bfcode{req\_get\_news}}{\emph{news\_url}}{}
get a Request object
\begin{quote}\begin{description}
\item[{Parameters}] \leavevmode
\textbf{news\_url} -- an url of a news like: {[}\href{http://news2.sysu.edu.cn/news01/141185.htm}{http://news2.sysu.edu.cn/news01/141185.htm}{]}

\item[{Returns}] \leavevmode
\emph{Request} - a Request object to open a news url

\end{description}\end{quote}

\end{fulllineitems}

\index{req\_get\_news\_urls() (in module sysunews.post\_module)}

\begin{fulllineitems}
\phantomsection\label{sysunews:sysunews.post_module.req_get_news_urls}\pysiglinewithargsret{\code{sysunews.post\_module.}\bfcode{req\_get\_news\_urls}}{\emph{module}, \emph{index='`}}{}
get a request object
\begin{quote}\begin{description}
\item[{Parameters}] \leavevmode\begin{itemize}
\item {} 
\textbf{module} -- 1  - 中大新闻 2  - 每周聚焦 3  - 媒体中大

\item {} 
\textbf{index} -- certain page of a module

\end{itemize}

\item[{Returns}] \leavevmode
\emph{Request} - a Request object

\end{description}\end{quote}

\end{fulllineitems}



\section{sysunews.server module}
\label{sysunews:module-sysunews.server}\label{sysunews:sysunews-server-module}\index{sysunews.server (module)}
server module

This module help to run a server program to solve HTTP get request from client base on webpy framwork
\index{Getnews (class in sysunews.server)}

\begin{fulllineitems}
\phantomsection\label{sysunews:sysunews.server.Getnews}\pysigline{\strong{class }\code{sysunews.server.}\bfcode{Getnews}}
Class to solve url /news
\begin{description}
\item[{Can solve url with three parameters}] \leavevmode
/news?module=1\&start=1\&num=1

\end{description}
\index{GET() (sysunews.server.Getnews method)}

\begin{fulllineitems}
\phantomsection\label{sysunews:sysunews.server.Getnews.GET}\pysiglinewithargsret{\bfcode{GET}}{}{}
Get news as json object
\begin{quote}\begin{description}
\item[{Returns}] \leavevmode
\emph{newslist} - a list of news in json type

\end{description}\end{quote}

\end{fulllineitems}


\end{fulllineitems}

\index{GetnewsNum (class in sysunews.server)}

\begin{fulllineitems}
\phantomsection\label{sysunews:sysunews.server.GetnewsNum}\pysigline{\strong{class }\code{sysunews.server.}\bfcode{GetnewsNum}}
Class to solve url /able
\begin{description}
\item[{Request url must contains a parameter}] \leavevmode
/able?module=1

\end{description}
\index{GET() (sysunews.server.GetnewsNum method)}

\begin{fulllineitems}
\phantomsection\label{sysunews:sysunews.server.GetnewsNum.GET}\pysiglinewithargsret{\bfcode{GET}}{}{}
get the num of news is certain module
\begin{quote}\begin{description}
\item[{Returns}] \leavevmode
(json) \emph{key} - count \& \emph{value} - the number of news of certain module in database:

\end{description}\end{quote}

\end{fulllineitems}


\end{fulllineitems}

\index{Getnews\_html (class in sysunews.server)}

\begin{fulllineitems}
\phantomsection\label{sysunews:sysunews.server.Getnews_html}\pysigline{\strong{class }\code{sysunews.server.}\bfcode{Getnews\_html}}
Class to solve url /news
\begin{description}
\item[{Can solve url with three parameters::}] \leavevmode
/news?module=1\&start=1\&num=1

\end{description}
\index{GET() (sysunews.server.Getnews\_html method)}

\begin{fulllineitems}
\phantomsection\label{sysunews:sysunews.server.Getnews_html.GET}\pysiglinewithargsret{\bfcode{GET}}{}{}
get html content of news
\begin{quote}\begin{description}
\item[{Returns}] \leavevmode
\emph{htmltext} - html contents of news that being request, can display in a pretty way in browser

\end{description}\end{quote}

\end{fulllineitems}


\end{fulllineitems}

\index{Images (class in sysunews.server)}

\begin{fulllineitems}
\phantomsection\label{sysunews:sysunews.server.Images}\pysigline{\strong{class }\code{sysunews.server.}\bfcode{Images}}
Handler that solve url /home/images/*
\index{GET() (sysunews.server.Images method)}

\begin{fulllineitems}
\phantomsection\label{sysunews:sysunews.server.Images.GET}\pysiglinewithargsret{\bfcode{GET}}{\emph{url}}{}
Get a images with a url

the url is the path to a images that has been saved in local directory /home/images/
\begin{quote}\begin{description}
\item[{Parameters}] \leavevmode
\textbf{url} -- image url like {[}/home/images/content/...{]}

\end{description}\end{quote}

\end{fulllineitems}


\end{fulllineitems}

\index{update\_news\_intime() (in module sysunews.server)}

\begin{fulllineitems}
\phantomsection\label{sysunews:sysunews.server.update_news_intime}\pysiglinewithargsret{\code{sysunews.server.}\bfcode{update\_news\_intime}}{\emph{minutes}}{}
update the news database
\begin{quote}\begin{description}
\item[{Parameters}] \leavevmode
\textbf{minutes} -- update the database every 10 minutes when minutes = 10

\end{description}\end{quote}

\end{fulllineitems}



\section{Module contents}
\label{sysunews:module-contents}\label{sysunews:module-sysunews}\index{sysunews (module)}

\chapter{Indices and tables}
\label{index:indices-and-tables}\begin{itemize}
\item {} 
\emph{genindex}

\item {} 
\emph{modindex}

\item {} 
\emph{search}

\end{itemize}


\renewcommand{\indexname}{Python Module Index}
\begin{theindex}
\def\bigletter#1{{\Large\sffamily#1}\nopagebreak\vspace{1mm}}
\bigletter{s}
\item {\texttt{sysunews}}, \pageref{sysunews:module-sysunews}
\item {\texttt{sysunews.api}}, \pageref{sysunews:module-sysunews.api}
\item {\texttt{sysunews.db\_module}}, \pageref{sysunews:module-sysunews.db_module}
\item {\texttt{sysunews.db\_update}}, \pageref{sysunews:module-sysunews.db_update}
\item {\texttt{sysunews.html\_extracting}}, \pageref{sysunews:module-sysunews.html_extracting}
\item {\texttt{sysunews.initial}}, \pageref{sysunews:module-sysunews.initial}
\item {\texttt{sysunews.post\_module}}, \pageref{sysunews:module-sysunews.post_module}
\item {\texttt{sysunews.server}}, \pageref{sysunews:module-sysunews.server}
\end{theindex}

\renewcommand{\indexname}{Index}
\printindex
\end{document}
